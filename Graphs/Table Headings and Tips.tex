  \documentclass[dvipsnames,final]{beamer} % beamer 3.10: do NOT use option hyperref={pdfpagelabels=false} !  
 %\documentclass[final,hyperref={pdfpagelabels=false}]{beamer} % beamer 3.07: get rid of beamer warnings
  	\mode<presentation> {  %% check http://www-i6.informatik.rwth-aachen.de/~dreuw/latexbeamerposter.php for examples
	\usetheme{default}    %% you should define your own theme e.g. for big headlines using your own logos 
	\beamertemplatenavigationsymbolsempty
	\definecolor{royalblue}{rgb}{0,0.13725490196078433,0.4}
	\definecolor{royalblueweb}{rgb}{0.25490196078431371,0.41176470588235292,0.88235294117647056}  
  	\definecolor{burntorange}{rgb}{0.8,0.3333333333333333,0}
  	\colorlet{description}{RoyalBlue} 
  	\colorlet{unit}{ForestGreen} 
  	\colorlet{symbol}{red}
  	\setbeamercolor{frametitle}{fg=RoyalBlue}
    	\setbeamertemplate{frametitle} {
	\begin{centering} 
    		\textbf{\insertframetitle} \par
	\end{centering}}
	\setbeamertemplate{itemize item}{\small\raise20pt\hbox{\donotcoloroutermaths$\blacktriangleright$}}
}
	
	\usepackage{fontspec}% provides font selecting commands 
	
	\defaultfontfeatures{Mapping=tex-text} % fix TeX quotes
	\usepackage{xunicode}% provides unicode character macros 
	\usepackage{xltxtra} % provides some fixes/extras 
	\setmainfont{Gentium Book Basic}
	\setsansfont{Arial}
	\setmonofont{Courier New}
	\usepackage[english]{babel}
  \usepackage{amsmath,amsthm, amssymb, latexsym}
  \usefonttheme{serif}
  \usepackage[orientation=landscape,size=letter,scale=4]{beamerposter}
  %\usepackage[orientation=portrait,size=a0,scale=1.4,debug]{beamerposter}                       % e.g. for DIN-A0 poster
  %\usepackage[orientation=portrait,size=a1,scale=1.4,grid,debug]{beamerposter}                  % e.g. for DIN-A1 poster, with optional grid and debug output
  %\usepackage[size=custom,width=45.72,height=60.96,scale=1.8,debug]{beamerposter}                     % e.g. for custom size poster (18in x 24in w/ printable 17in x 23in)
  %\usepackage[orientation=portrait,size=a0,scale=1.0,printer=rwth-glossy-uv.df]{beamerposter}   % e.g. for DIN-A0 poster with rwth-glossy-uv printer check
  % ...
  %
  \geometry{margin=.5in}
  \usepackage{tikz}
 	\usepackage{booktabs}
  
  \title[Tables]{Table Headings and Tips}
  \author[Vancil]{Brian Vancil}
  \institute[Sumner]{Sumner Academy of Arts & Science}
  \date{2013-10-06}
  
  \begin{document}
  \begin{frame}{Raw Data Column}
  %\vspace{-2cm}
	\begin{center}
		\Huge
		\begin{tabular}{@{}l@{}} \toprule[10pt]
			\multicolumn{1}{c}{\textcolor{description}{Raw Variable Description}} \\
			\multicolumn{1}{c}{\textcolor{symbol}{\textit{Symbol}}/\textcolor{unit}{Unit}} \\
			\multicolumn{1}{l}{$\pm$ Uncertainty}\\ 				\midrule[5pt]
		\end{tabular}
		\Large
		\begin{itemize}
			\item Write all values with the same precision.
			\item For uncertainty, use the least count or half the range of repeated measurements, whichever is larger.
		\end{itemize}
	\end{center}
	\end{frame}
  \begin{frame}{Processed Data Column}
  %\vspace{-2cm}
	\begin{center}
		\Huge
		\begin{tabular}{@{}l@{}} \toprule[10pt]
			\multicolumn{1}{c}{\textcolor{description}{Processed Variable Description}} \\
			\multicolumn{1}{c}{\textcolor{symbol}{\textit{Symbol}}/\textcolor{unit}{Unit}} \\
			\multicolumn{1}{c}{$\textcolor{symbol}{\textit{Symbol}}=\textit{Formula}$} \\
			\midrule[5pt]
		\end{tabular}
		\Large		
		\begin{itemize}
			\item Write all values with the same precision.
			\item Write out all steps of calculation for one of the data points (rows).  For the rest, just follow the same pattern and use a calculator or spreadsheet.
			\item Processed uncertainty is optional.  You can try by calculating its largest and smallest possible values.
		\end{itemize}
	\end{center}
  \end{frame}

\end{document}
