% XeLaTeX can use any Mac OS X font. See the setromanfont command below.
% Input to XeLaTeX is full Unicode, so Unicode characters can be typed directly into the source.

% The next lines tell TeXShop to typeset with xelatex, and to open and save the source with Unicode encoding.

%!TEX TS-program = xelatex
%!TEX encoding = UTF-8 Unicode

\documentclass[12pt]{article}
\usepackage[margin=0.5in]{geometry}                % See geometry.pdf to learn the layout options. There are lots.
\geometry{letterpaper}                   % ... or a4paper or a5paper or ... 
%\geometry{landscape}                % Activate for for rotated page geometry
%\usepackage[parfill]{parskip}    % Activate to begin paragraphs with an empty line rather than an indent
\usepackage{graphicx}
\usepackage{amssymb}

\usepackage{array}
\usepackage{booktabs}

% Will Robertson's fontspec.sty can be used to simplify font choices.
% To experiment, open /Applications/Font Book to examine the fonts provided on Mac OS X,
% and change "Hoefler Text" to any of these choices.

\usepackage{fontspec,xltxtra,xunicode}
\defaultfontfeatures{Mapping=tex-text}
\setromanfont[Mapping=tex-text]{Baskerville} % Hoefler Text has terrible numbers
\setsansfont[Scale=MatchLowercase,Mapping=tex-text]{Gill Sans}
\setmonofont[Scale=MatchLowercase]{Andale Mono}

\title{Motion Terms and Symbols \\ for Uniform Motion and Uniform Acceleration }
%\author{Brian Vancil}
\date{}                                           % Activate to display a given date or no date

\begin{document}
\maketitle
\vspace{-2cm}

% For many users, the previous commands will be enough.
% If you want to directly input Unicode, add an Input Menu or Keyboard to the menu bar 
% using the International Panel in System Preferences.
% Unicode must be typeset using a font containing the appropriate characters.
% Remove the comment signs below for examples.

% \newfontfamily{\A}{Geeza Pro}
% \newfontfamily{\H}[Scale=0.9]{Lucida Grande}
% \newfontfamily{\J}[Scale=0.85]{Osaka}

% Here are some multilingual Unicode fonts: this is Arabic text: {\A السلام عليكم}, this is Hebrew: {\H שלום}, 
% and here's some Japanese: {\J 今日は}.

\begin{center}
How do we describe motion in time and space?
  \begin{tabular}{@{} l>{\raggedright\arraybackslash}p{6cm}cc @{}}
    \toprule
    term & description & 1-D symbol & vector symbol \\ \hline 
    \midrule
    motion & description of position of a particle at all times & $x(t)$ & $\vec{x}(t)$ \\ \hline
    frame of reference & coordinate system (choice of positive direction and measurement unit) used to describe motion & & \\ \hline
    
    clock reading & time read from a clock & t & t \\ \hline 
    time interval & time elapsed from initial to final measurement & $\Delta t$ & $\Delta t$ \\ \hline 
    position & location in space measured on a coordinate system & $x$ & $\vec{x}$ \\ \hline 
    displacement & change in position from initial to final measurement & $\Delta x$ & $\Delta\vec{x}$ \\ \hline 
    distance & length of the shortest path between two positions & $d$ & $d$ \\ \hline 
    average velocity & rate of displacement per unit time interval from initial to final measurements & $\overline{v}$ & $\overline{\vec{v}}$ \\ \hline 
    instantaneous velocity & average velocity over a very short time interval & $v$ & $\vec{v}$ \\ \hline 
    average speed & rate of distance per unit time interval &  $\overline{|v|}$ & $\overline{v}$ \\ \hline
    instantaneous speed & average speed over a very short time interval; magnitude (absolute value) of velocity;  & $|v|$ & $v$ \\ \hline
    change in velocity & change in velocity from initial to final values & $\Delta v$ & $\Delta\vec{v}$ \\ \hline 
    average acceleration & rate of velocity change per unit time interval from initial to final measurements & $\overline{a}$ & $\overline{\vec{a}}$ \\ \hline 
    instantaneous acceleration & average acceleration over a very short time interval & $a$ & $\vec{a}$ \\
    \bottomrule
  \end{tabular}

Compare to Chapter 11, sections 1 and 2 (pages 362-379) of the textbook.
\end{center}

\end{document}  