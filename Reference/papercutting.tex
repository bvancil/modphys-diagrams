% XeLaTeX can use any Mac OS X font. See the setromanfont command below.
% Input to XeLaTeX is full Unicode, so Unicode characters can be typed directly into the source.

% The next lines tell TeXShop to typeset with xelatex, and to open and save the source with Unicode encoding.

%!TEX TS-program = xelatex
%!TEX encoding = UTF-8 Unicode

\documentclass[12pt]{article}
\usepackage{geometry}                % See geometry.pdf to learn the layout options. There are lots.
\geometry{letterpaper}                   % ... or a4paper or a5paper or ... 
\geometry{margin=1cm}
%\geometry{landscape}                % Activate for for rotated page geometry
%\usepackage[parfill]{parskip}    % Activate to begin paragraphs with an empty line rather than an indent
\usepackage{graphicx}
\usepackage{amssymb}
\usepackage{tikz}

% Will Robertson's fontspec.sty can be used to simplify font choices.
% To experiment, open /Applications/Font Book to examine the fonts provided on Mac OS X,
% and change "Hoefler Text" to any of these choices.

\usepackage{fontspec,xltxtra,xunicode}
\defaultfontfeatures{Mapping=tex-text}
\setromanfont[Mapping=tex-text]{Didot}
\setsansfont[Scale=MatchLowercase,Mapping=tex-text]{Gill Sans}
\setmonofont[Scale=MatchLowercase]{Andale Mono}

\title{Cutting Paper Evenly: An Exercise in Anal-Retentiveness}
\author{Brian Vancil}
%\date{}                                           % Activate to display a given date or no date

\newcommand{\equals}{=}

\begin{document}
\maketitle
The following are diagrams for cutting $8\frac12''\times11''$ paper evenly.
% Used in sage:
%def mixed(x):
%	whole = floor(x)
%	part = x-whole
%	return latex(whole)+latex(part)
%	
%def measure(x):
%	whole = floor(x)
%	part = 16*(x-whole)
%	if part <= 1: 
%		part = latex(part)
%	else: 
%		part = mixed(part)
%	return latex(whole)+'\\frac{'+part+'}{16}}'
	
\begin{tikzpicture}[scale=0.7] % divide in 2
\draw[xstep=4.24,ystep=5.49] (0,0) grid (8.5,11);
\node at (4.25,0) [label=below:$4\frac14''$] {};
\node at (0,5.5) [label=left:$5\frac12''$] {};
\end{tikzpicture}\hspace{1cm}
\begin{tikzpicture}[scale=0.7] % divide in 3
\draw[xstep=2.83,ystep=3.66] (0,0) grid (8.5,11);
\node at (2.83333333333333,0) [label=below:$2\frac{13\frac{1}{3}}{16}''$] {};
\node at (5.66666666666667,0) [label=below:$5 \frac{ 10 \frac{2}{3} }{16}''$] {};
\node at (0,3.66666666666667) [label=left:$3 \frac{ 10 \frac{2}{3} }{16}''$] {};
\node at (0,7.33333333333333) [label=left:$7 \frac{ 5 \frac{1}{3} }{16}''$] {};
\end{tikzpicture}\vspace{1cm}

\begin{tikzpicture}[scale=0.7] % divide in 4
\draw[xstep=2.124,ystep=2.74] (0,0) grid (8.5,11);
\node at (2.125,0) [label=below:$2\frac14''$] {};
\node at (4.25,0) [label=below:$4\frac14''$] {};
\node at (6.375,0) [label=below:$6\frac38''$] {};
\node at (0,2.75) [label=left:$2\frac34''$] {};
\node at (0,5.5) [label=left:$5\frac12''$] {};
\node at (0,8.25) [label=left:$8\frac14''$] {};
\end{tikzpicture}
\end{document}  