  \documentclass[final]{beamer} % beamer 3.10: do NOT use option hyperref={pdfpagelabels=false} !
 %\documentclass[final,hyperref={pdfpagelabels=false}]{beamer} % beamer 3.07: get rid of beamer warnings
  	\mode<presentation> {  %% check http://www-i6.informatik.rwth-aachen.de/~dreuw/latexbeamerposter.php for examples
	\usetheme{default}    %% you should define your own theme e.g. for big headlines using your own logos 
	\beamertemplatenavigationsymbolsempty
	\definecolor{royalblue}{rgb}{0,0.13725490196078433,0.4}
	\definecolor{royalblueweb}{rgb}{0.25490196078431371,0.41176470588235292,0.88235294117647056}  
  	\definecolor{burntorange}{rgb}{0.8,0.3333333333333333,0}
	\definecolor{silver}{rgb}{0.75294117647058822,0.75294117647058822,0.75294117647058822}
  	\setbeamercolor{frametitle}{fg=blue!80!black}
    	\setbeamertemplate{frametitle} {
	\begin{center} 
    		\vspace{-2.5cm}\textbf{\insertframetitle} \par 
		\normalsize\textbf{\insertframesubtitle}
	\end{center}
	}
	\setbeamertemplate{enumerate items}[circle]
	\setbeamercolor{structure}{fg=burntorange}
	\setbeamercolor{enumerate item projected}{fg=white}
	\setbeamerfont{item projected}{size=\normalsize}
	\setbeamertemplate{enumerate item}
		{
		  \usebeamerfont*{item projected}%
		  \usebeamercolor[bg]{item projected}%
		  \begin{pgfpicture}{-1ex}{0ex}{1ex}{2ex}
		    \pgfpathcircle{\pgfpoint{0pt}{.75ex}}{1.2ex}
		    \pgfusepath{fill}
		    \pgftext[base]{\color{fg}\insertenumlabel}
		  \end{pgfpicture}%
		}
	
  }
  \usepackage[english]{babel}
  \usepackage[latin1]{inputenc}
  \usepackage{amsmath,amsthm, amssymb, latexsym}
  \usepackage{bbding}
  %\usepackage{times}
  %\usefonttheme{professionalfonts}  % times is obsolete
  \usefonttheme[onlymath]{serif}
  \boldmath
  %\usepackage[orientation=portrait,size=a0,scale=1.4,debug]{beamerposter}                       % e.g. for DIN-A0 poster
  %\usepackage[orientation=portrait,size=a1,scale=1.4,grid,debug]{beamerposter}                  % e.g. for DIN-A1 poster, with optional grid and debug output
  \usepackage[size=custom,width=45.72,height=60.96,scale=1.8,debug]{beamerposter}                     % e.g. for custom size poster (18in x 24in w/ printable 17in x 23in)
  %\usepackage[orientation=portrait,size=a0,scale=1.0,printer=rwth-glossy-uv.df]{beamerposter}   % e.g. for DIN-A0 poster with rwth-glossy-uv printer check
  % ...
  %
  \geometry{margin=1in}
  \def\imagetop#1{\vtop{\vspace{-1.5cm}\null\hbox{#1}\vspace{-1.5cm}}}
  \usepackage{tikz}
  
  \newcommand{\header}[1]{\textcolor{royalblueweb}{\textbf{#1}}}
  \newcommand{\spacing}{\vspace{1.0em}}
    \newcommand{\parasep}{\vspace{-0.0\baselineskip}\textcolor{silver}{\large\hfill\FiveStar\hfill\FiveStar\hfill\FiveStar\hfill}\vspace{-0.0\baselineskip}}
    
  % From Andrew Smith <andrew.smith@ASD20.ORG> MODELING PHYSICS LISTSERVE
  \title[Rules of Ten]{Rules of Ten: rules of thumb for data collection and processing}
  \author[Vancil]{Brian Vancil}
  \institute[Sumner]{Sumner Academy of Arts & Sciences}
  \date{2012-04-07}
  
  \begin{document}
  \begin{frame}{Rules of Ten} 
  \framesubtitle{rules of thumb for data collection and processing}
  \vspace{-1.5em}\parasep
	\begin{enumerate}
	
\item \header{Collect at least 10 data points.}  A data point is one pairing of independent and dependent variable measurements.  Without enough data points, we cannot reliably find trends in the data. \spacing

\item \header{The largest independent variable measurement should be at least 10 times the smallest independent variable measurement.}  Nature sometimes surprises us at larger or smaller scales than we think to look. \spacing

\item \header{We like the uncertainty in our measurements to be less than 10\% of the range of the measurements.}  There is no point in trying to understand our results mathematically if the variation we see is around the same size as the uncertainty in the measurements. \spacing

\item \header{We like the root mean square error (RMSE) for a fit to be less than 10\% of the range of dependent variable measurements.}  A large RMSE means that our mathematical model does not fit the data very well. \spacing

\item \header{We will consider the vertical intercept negligible if it is less than 5\% of the range of the dependent variable measurements.}  A vertical intercept is likely to be meaningful if it has a decent magnitude compared with our dependent variable measurements.

	\end{enumerate}
        \spacing\parasep
        %\vfill
	\begin{center}\footnotesize
	 Adapted from a list by Andrew~Smith
	\end{center}
  \end{frame}
  \end{document}