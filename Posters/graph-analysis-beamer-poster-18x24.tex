  \documentclass[final]{beamer} % beamer 3.10: do NOT use option hyperref={pdfpagelabels=false} !
 %\documentclass[final,hyperref={pdfpagelabels=false}]{beamer} % beamer 3.07: get rid of beamer warnings
  	\mode<presentation> {  %% check http://www-i6.informatik.rwth-aachen.de/~dreuw/latexbeamerposter.php for examples
	\usetheme{default}    %% you should define your own theme e.g. for big headlines using your own logos 
	\beamertemplatenavigationsymbolsempty
	\definecolor{royalblue}{rgb}{0,0.13725490196078433,0.4}
	\definecolor{royalblueweb}{rgb}{0.25490196078431371,0.41176470588235292,0.88235294117647056}  
  	\definecolor{burntorange}{rgb}{0.8,0.3333333333333333,0}
  	\setbeamercolor{frametitle}{fg=blue!80!black}
    	\setbeamertemplate{frametitle} {
	\begin{centering} 
    		\vspace{-1.5cm}\textbf{\insertframetitle} \par
	\end{centering}
}
  }
  \usepackage[english]{babel}
  \usepackage[latin1]{inputenc}
  \usepackage{amsmath,amsthm, amssymb, latexsym}
  %\usepackage{times}\usefonttheme{professionalfonts}  % times is obsolete
  \usefonttheme[onlymath]{serif}
  \boldmath
  %\usepackage[orientation=portrait,size=a0,scale=1.4,debug]{beamerposter}                       % e.g. for DIN-A0 poster
  %\usepackage[orientation=portrait,size=a1,scale=1.4,grid,debug]{beamerposter}                  % e.g. for DIN-A1 poster, with optional grid and debug output
  \usepackage[size=custom,width=45.72,height=60.96,scale=1.8,debug]{beamerposter}                     % e.g. for custom size poster (18in x 24in w/ printable 17in x 23in)
  %\usepackage[orientation=portrait,size=a0,scale=1.0,printer=rwth-glossy-uv.df]{beamerposter}   % e.g. for DIN-A0 poster with rwth-glossy-uv printer check
  % ...
  %
  \geometry{margin=.5in}
  \usepackage{array}
  \usepackage{booktabs}
  \newcolumntype{P}[1]{>{\raggedright\large}p{#1}}
  \def\imagetop#1{\vtop{\vspace{-1.5cm}\null\hbox{#1}\vspace{-1.5cm}}}
  \usepackage{tikz}
  
  \newcommand{\xx}{\textcolor{variable}{x}}
  \newcommand{\yy}{\textcolor{variable}{y}}
  \newcommand{\versus}{vs\ }
  \newcommand{\plotscale}{1.5}
  \newcommand{\plotline}{6pt}
  \newcommand{\formatmm}[1]{\textcolor{royalblue}{\textbf{#1}}}
  \colorlet{plot}{burntorange}
  \colorlet{variable}{blue!80!black}
  
  \title[Graph Methods]{Graphical Methods Summary}
  \author[Vancil]{Brian Vancil}
  \institute[Sumner]{Sumner Academy of Arts & Science}
  \date{2012-04-07}
  
  \begin{document}
  \begin{frame}{Graphical Methods Summary} 
    \defaultaddspace=.25em
    \vspace{-2cm}
    \begin{tabular}{P{.25\linewidth}P{.16\linewidth}P{.30\linewidth}@{\quad}>{\arraybackslash}P{.21\linewidth}} \toprule[.1em]
    \normalsize Mathematical model & \normalsize Graph shape & \normalsize Written relationship & \normalsize To linearize, graph\ldots \\ \midrule[.1em] \addlinespace
    
    \formatmm{constant}\ \ \  $\yy=b$ & 
    \imagetop{\begin{tikzpicture}[scale=\plotscale,domain=0:4,line width=\plotline,smooth]
    \draw[color=plot] plot (\x,3);
    \draw[<->] (0,4) -- (0,0) -- (4,0); 
    \end{tikzpicture}}
    & $\yy$ is constant. & \\ \addlinespace \midrule \addlinespace
    
    \formatmm{proportional} $\yy=m\xx$ &
    \imagetop{\begin{tikzpicture}[scale=\plotscale,domain=0:4,line width=\plotline,smooth]
    \draw[color=plot] plot (\x,.8*\x);
    \draw[<->] (0,4) -- (0,0) -- (4,0); 
    \end{tikzpicture}}    
    & $\yy$ is directly proportional to $\xx$.  & \\ \addlinespace  \midrule \addlinespace

    \formatmm{linear} $\yy=m\xx+b$ & 
    \imagetop{\begin{tikzpicture}[scale=\plotscale,domain=0:4,line width=\plotline,smooth,samples=40]
    \draw[color=plot] plot (\x,.6*\x+1);
    \draw[<->] (0,4) -- (0,0) -- (4,0); 
    \end{tikzpicture}}    
    & $\yy$ is linear in $\xx$.  & \\ \addlinespace \midrule \addlinespace
    
    \formatmm{inversely proportional} $\yy=\frac{a}{\xx}$ & 
    \imagetop{\begin{tikzpicture}[scale=\plotscale,domain=.1:4,line width=\plotline,smooth,samples=40]
    \draw[color=plot] plot (\x,.4/\x);
    \draw[<->] (0,4) -- (0,0) -- (4,0); 
    \end{tikzpicture}}    
    & $\yy$ is inversely proportional to $\xx$.  & $\yy$ \versus $\frac{1}{\xx}$ \\ \addlinespace \midrule \addlinespace
    
    \formatmm{power law} $\yy=a\xx^{n}$ & 
    \imagetop{\begin{tikzpicture}[scale=\plotscale,domain=0:4,line width=\plotline,smooth,samples=40]
    \draw[color=plot] plot (\x,.25*\x*\x);
    \draw[<->] (0,4) -- (0,0) -- (4,0); 
    \end{tikzpicture}}    
    & $\yy$ is proportional to $\xx^{n}$.  & $\yy$ \versus $\xx^{n}$ \\ \addlinespace \midrule \addlinespace
    
    \formatmm{square root} $\yy=a\sqrt{\xx}$ &  
    \imagetop{\begin{tikzpicture}[scale=\plotscale,domain=0:4,line width=\plotline,smooth,samples=40]
    \draw[color=plot] plot (.25*\x*\x,\x);
    \draw[<->] (0,4) -- (0,0) -- (4,0); 
    \end{tikzpicture}}    
    %& $\yy^{2}$ is proportional to $\xx$. & Graph $\yy^{2}$ \versus $\xx$ \\ \addlinespace \midrule \addlinespace
    & $\yy$ is proportional to the square root of $\xx$. & $\yy^{2}$ \versus $\xx$ \\ \addlinespace \midrule \addlinespace
    
    \formatmm{exponential} $\yy=ab^{\xx}$ & 
    \imagetop{\begin{tikzpicture}[scale=\plotscale,domain=0:4,line width=\plotline,smooth,samples=40]
    \draw[color=plot] plot (\x,{pow(pow(4,.25),\x)});
    \draw[<->] (0,4) -- (0,0) -- (4,0); 
    \end{tikzpicture}}    
    & $\yy$ is exponential in $\xx$. & $\log \yy$ \versus $\xx$ \\  \addlinespace
    
    \bottomrule[.1em]
    \end{tabular}
  \end{frame}
  \end{document}