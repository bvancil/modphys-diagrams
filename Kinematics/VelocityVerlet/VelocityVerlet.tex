\documentclass[t]{beamer}
%\documentclass[t,handout]{beamer}
%\usepackage{pgfpages} 
%\pgfpagesuselayout{2 on 1}[letterpaper,border shrink=5mm]
%\nofiles

%\documentclass{article}
%\usepackage{beamerarticle}

\mode<presentation>
{
  \usetheme{Sumner}
  \setbeamercovered{invisible}
}

%\usepackage[letterpaper,margin=0.5in]{geometry} 
%\usepackage[parfill]{parskip}    % Activate to begin paragraphs with an empty line rather than an indent
\usepackage{graphicx}
\usepackage{amssymb}
\usepackage{amsmath}
\usepackage{tikz}
%\usepackage{asymptote}
%\usepackage{array}
%\usepackage{longtable}

\usepackage{fontspec,xltxtra,xunicode}
\defaultfontfeatures{Mapping=tex-text}
%\setromanfont[Mapping=tex-text]{Baskerville}
%\setsansfont[Scale=MatchLowercase,Mapping=tex-text]{Gill Sans}
%\setmonofont[Scale=MatchLowercase]{Andale Mono}

%\def\imagetop#1{\vtop{\null\hbox{#1}}}

\setlength{\abovedisplayskip}{0pt}
\setlength{\belowdisplayskip}{0pt}

\newcommand{\unit}[1]{\ \text{#1}}
\newcommand{\correct}{}
\newcommand{\question}[6]{
\begin{frame}{#1}
	\renewcommand{\correct}{}
	\large
	#2
	\begin{enumerate}[a]
	\item {#3}
	\item {#4}
	\item {#5}
	\item {#6}
	\end{enumerate}
\end{frame}
\begin{frame}{#1 (answer)}
	\renewcommand{\correct}{\bf (\textcolor{active}{\checkmark}) }
	\large
	#2
	\begin{enumerate}[a]
	\item {#3}
	\item {#4}
	\item {#5}
	\item {#6}
	\end{enumerate}
\end{frame}
}
\newcommand{\header}[1]{\textbf{\color{royalblueweb}{#1}}}

% convenience functions
\newcommand{\dt}{\Delta t}
\newcommand{\tp}{t+\Delta t} % t' read t-prime
\newcommand{\later}[1]{\textcolor{royalblueweb}{#1}}
\newcommand{\now}[1]{\textcolor{burntorange}{#1}}

\title{When force isn't constant}
\subtitle{Or, the Velocity Verlet method}
\author{Mr.~B.~Vancil}
\institute[Sumner]
{
	Science Department\\
	Sumner Academy of Arts \& Sciences
}
\date{}                                           % Activate to display a given date or no date

\subject{Physics}

\begin{document}

\begin{frame}
  \titlepage
\end{frame}

\begin{frame}{When force isn't constant} \Large \vspace{-0.5cm}
	We can't use our constant acceleration (CAPM) formulas!  Why? 
	\pause
	
	So, we will chop our {\LARGE BIG time interval} into a lot of {\normalsize very, very small time intervals} and assume that the force changes slowly enough so acceleration is approximately constant over each {\normalsize very, very small time interval}. 
	\begin{tikzpicture}[thick]
  		\draw(0,0) -- (8,0) node[midway,above] {\LARGE BIG time interval};
		\draw(0,0) -- (0, 3pt) node[above] {0 seconds};
		\draw(8,0) -- (8, 3pt) node[above] {final clock reading};
		\foreach \x in {0,.25,...,8}
			\draw(\x,0pt)--(\x,-3pt);
		\draw (0,-6pt)--(.25,-6pt) node[anchor=north west] {\normalsize very, very small time intervals};
	 \end{tikzpicture}\pause
	
	Summary of notation:\normalsize
	
	\begin{tabular}{c|c}
	now (clock reading $\now{t}$) & a bit later (clock reading $\later{\tp}$) \\\hline
	$\now{x(t)}$, $\now{v(t)}$, $\now{a(t)}$ & $\later{x(\tp)}$, $\later{v(\tp)}$, $\later{a(\tp)}$
	\end{tabular}
\end{frame}

\begin{frame}{Velocity Verlet Method}\large\vspace{-0.5cm}
It's not quite good enough to assume acceleration is constant.  When we calculate the new position, we \emph{will} use our standard
$$\later{x(\tp)}=\now{x(t)}+\now{v(t)}\dt+\frac{1}{2}\now{a(t)}\left(\dt\right)^{2},$$
but when we calculate the new velocity, we will use the average of the old and new velocities, 
$$\later{v(\tp)}=\now{v(t)}+\frac{\now{a(t)}+\later{a(\tp)}}{2}\dt.$$
Of course, to calculate acceleration, we will have to use Newton's 2nd Law:
\begin{align*}
\now{a(t)}&=\frac{\now{F(x(t),t)}}{m} \\
\later{a(\tp)}&=\frac{\later{F(x(\tp),\tp)}}{m}
\end{align*}	
\end{frame}

\begin{frame}{Velocity Verlet Method algorithm}
	\begin{enumerate}
	\item Set the initial clock reading $t=0$, initial position $x(0)$ and velocity $v(0)$.
	\item Calculate initial acceleration from N2L: $a(0)=\frac{F(x(0),0)}{m}$.
	\item Choose a time step $\dt$.
	\item Calculate the next position from $\later{x(\tp)}=\now{x(t)}+\now{v(t)}\dt+\frac{1}{2}\now{a(t)}\left(\dt\right)^{2}$.
	\item Calculate the next acceleration from $\later{a(\tp)}=\frac{\later{F(x(\tp),\tp)}}{m}$.
	\item Calculate the new velocity from $\later{v(\tp)}=\now{v(t)}+\frac{\now{a(t)}+\later{a(\tp)}}{2}\dt.$
	\item Add $\dt$ to $t$.
	\item Repeat steps 4-7 as long as desired.
	\item Decrease $\dt$ and repeat steps 3-8 until your result seems reasonable and doesn't change too much each time you decrease it.
	\end{enumerate}
\end{frame}
\end{document}
