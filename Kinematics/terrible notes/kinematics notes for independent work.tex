\documentclass[t]{beamer}
%\documentclass[t,handout]{beamer}
%\usepackage{pgfpages} 
%\pgfpagesuselayout{2 on 1}[letterpaper,border shrink=5mm]
%\nofiles

%\documentclass{article}
%\usepackage{beamerarticle}

\mode<presentation>
{
  \usetheme{Sumner}
}

%\usepackage[letterpaper,margin=0.5in]{geometry} 
%\usepackage[parfill]{parskip}    % Activate to begin paragraphs with an empty line rather than an indent
\usepackage{graphicx}
\usepackage{amssymb}
\usepackage{amsmath}
\usepackage{tikz}
%\usepackage{asymptote}
%\usepackage{array}
%\usepackage{longtable}

\usepackage{fontspec,xltxtra,xunicode}
\defaultfontfeatures{Mapping=tex-text}
%\setromanfont[Mapping=tex-text]{Baskerville}
%\setsansfont[Scale=MatchLowercase,Mapping=tex-text]{Gill Sans}
%\setmonofont[Scale=MatchLowercase]{Andale Mono}

%\def\imagetop#1{\vtop{\null\hbox{#1}}}

\setlength{\abovedisplayskip}{0pt}
\setlength{\belowdisplayskip}{0pt}

\newcommand{\unit}[1]{\ \text{#1}}
\newcommand{\correct}{}
\newcommand{\question}[6]{
\begin{frame}{#1}
	\renewcommand{\correct}{}
	\large
	#2
	\begin{enumerate}[a]
	\item {#3}
	\item {#4}
	\item {#5}
	\item {#6}
	\end{enumerate}
\end{frame}
\begin{frame}{#1 (answer)}
	\renewcommand{\correct}{\bf (\textcolor{active}{\checkmark}) }
	\large
	#2
	\begin{enumerate}[a]
	\item {#3}
	\item {#4}
	\item {#5}
	\item {#6}
	\end{enumerate}
\end{frame}
}
\newcommand{\header}[1]{\textbf{\color{royalblueweb}{#1}}}

\title{Notes on kinematics for independent work}
\subtitle{in which you work with a partner to advance your understanding}
\author{Mr.~B.~Vancil}
\institute[Sumner]
{
	Science Department\\
	Sumner Academy of Arts \& Science
}
\date{Fall 2011}                                           % Activate to display a given date or no date

\subject{physics, kinematics}

\begin{document}

\begin{frame}
  \titlepage
\end{frame}

\begin{frame}{Summary: Measuring events}
\Large
\structure{Recall how we measure events:}
\begin{columns}[t]
\begin{column}{.45\linewidth}
\header{\underline{time}}
\begin{itemize}
\item clock reading {($t$)}
\item time interval {($\Delta t$)}
\end{itemize}
\end{column}
\begin{column}{.45\linewidth}
\header{\underline{length}}
\begin{itemize}
\item position {($x$)}
\item displacement {($\Delta x$)}
\item distance {($d$)}
\end{itemize}
\end{column}
\end{columns}
\structure{Discussion:}
Talk with your partner to\ldots
\begin{itemize}
\item define each term;
\item come up with a situation in which the measure of the clock reading is different from the measure of the time interval; and
\item come up with a situation in which the measure of position, displacement, and distance are all different.
\end{itemize}
\end{frame}

\begin{frame}{Average velocity versus average speed}
\Large
\structure{Recall how we measure events:}
\begin{columns}[t]
\begin{column}{.45\linewidth}
\header{\underline{time}}
\begin{itemize}
\item clock reading {($t$)}
\item time interval {($\Delta t$)}
\end{itemize}
\end{column}
\begin{column}{.45\linewidth}
\header{\underline{length}}
\begin{itemize}
\item position {($x$)}
\item displacement {($\Delta x$)}
\item distance {($d$)}
\end{itemize}
\end{column}
\end{columns}
\structure{Discussion:}\huge

Discuss with your partner: What is the difference between average velocity and average speed?
\end{frame}

\begin{frame}{Quizzes}\Large
\begin{enumerate}
\item Take ``\textbf{Unit 1: Uniform Motion Quiz 1b Terms and Unit Conversion}'' on your own now.  The last question is challenging, but try to answer it anyway.
\item When you finish, take ``\textbf{Unit 1: Graphing Time and Position Quiz 2}'' on your own.  You may not know what \emph{scalar} and \emph{vector} mean.  That's OK.  Pretend that you know, and guess.
\item When you finish both quizzes, write the question ``\textbf{What is an instant?}'' in your notes and answer using a complete sentence.  (You can change to the next slide before everyone is done with this.)
\end{enumerate}
\end{frame}

\begin{frame}{What does ``instantaneous'' mean?}\LARGE
Write the question ``\textbf{What is an instant?}'' in your notes and answer using a complete sentence.

Then, discuss your answer with your partner, trying to answer the following questions.
	\begin{itemize}
		\item How many seconds does a clock reading last?
		\item How long does an instant last?
		\item If an object is moving continuously, how long does it stay on any one position?
	\end{itemize}
\end{frame}

\begin{frame}{Measuring on a graph}
Given a ``position versus clock reading'' ($x$ v. $t$) graph, we want to be able to interpret the relationship between events.
\begin{columns}[T]
\begin{column}{.6\linewidth}
\begin{tikzpicture}[scale=0.5]
\draw[very thin, color=silver] (-.5,-2.2) grid (10.2,10.2);
\draw[->](-.5,0) -- (10.2,0) node[right] {$t\unit{(s)}$};
\draw[->](0,-2.2) -- (0, 10.2) node[above] {$x\unit{(m)}$};
\foreach \t in {0,1,...,10}
	\draw (\t,1pt) -- (\t,-1pt) node[anchor=north] {\t};
\foreach \x in {-2,-1,...,10}
	\draw (1pt,\x) -- (-1pt,\x) node[anchor=east] {\x};
\draw[thick,color=royalblueweb] plot[samples=4,domain=0:4] (\x,{10-3*(\x)});
\draw[thick,color=royalblueweb] plot[samples=4,domain=0:6] (\x+4,{-2+1*(\x)});
\fill[color=royalblueweb] (0,10) circle (3pt);
\draw[color=royalblueweb] (0,10) node[right] {A};
\fill[color=royalblueweb] (4,-2) circle (3pt);
\draw[color=royalblueweb] (4,-2) node[below] {B};
\fill[color=royalblueweb] (10,4) circle (3pt);
\draw[color=royalblueweb] (10,4) node[above] {C};
\end{tikzpicture} 
\end{column}
\begin{column}{.4\linewidth}\Large
Discuss with your partner:
\begin{enumerate}
\item Interpret point A.
\item Interpret point B.
\item Interpret point C.
\end{enumerate}
\end{column}
\end{columns}
\end{frame}

\begin{frame}{Measuring horizontally on a graph, 1}
\begin{columns}[T]
\begin{column}{.6\linewidth}
\begin{tikzpicture}[scale=0.5]
\draw[very thin, color=silver] (-.5,-2.2) grid (10.2,10.2);
\draw[->](-.5,0) -- (10.2,0) node[right] {$t\unit{(s)}$};
\draw[->](0,-2.2) -- (0, 10.2) node[above] {$x\unit{(m)}$};
\foreach \t in {0,1,...,10}
	\draw (\t,1pt) -- (\t,-1pt) node[anchor=north] {\t};
\foreach \x in {-2,-1,...,10}
	\draw (1pt,\x) -- (-1pt,\x) node[anchor=east] {\x};
\draw[thick,color=royalblueweb] plot[samples=4,domain=0:4] (\x,{10-3*(\x)});
\draw[thick,color=royalblueweb] plot[samples=4,domain=0:6] (\x+4,{-2+1*(\x)});
\fill[color=royalblueweb] (0,10) circle (3pt);
\draw[color=royalblueweb] (0,10) node[right] {A};
\fill[color=royalblueweb] (4,-2) circle (3pt);
\draw[color=royalblueweb] (4,-2) node[below] {B};
\fill[color=royalblueweb] (10,4) circle (3pt);
\draw[color=royalblueweb] (10,4) node[above] {C};
\draw[very thick,color=burntorange] (0,0) -- (10,0);
\draw[very thick,color=black,dashed] (0,10) -- (0,0);
\draw[very thick,color=black,dashed] (10,4) -- (10,0);
\end{tikzpicture} 
\end{column}
\begin{column}{.4\linewidth}
\LARGE
Discuss with your partner: What is the interpretation of the burnt orange horizontal line segment?  Does the length have a meaning?
\end{column}
\end{columns}
\end{frame}

\begin{frame}{Measuring horizontally on a graph, 2}
\begin{columns}[T]
\begin{column}{.6\linewidth}
\begin{tikzpicture}[scale=0.5]
\draw[very thin, color=silver] (-.5,-2.2) grid (10.2,10.2);
\draw[->](-.5,0) -- (10.2,0) node[right] {$t\unit{(s)}$};
\draw[->](0,-2.2) -- (0, 10.2) node[above] {$x\unit{(m)}$};
\foreach \t in {0,1,...,10}
	\draw (\t,1pt) -- (\t,-1pt) node[anchor=north] {\t};
\foreach \x in {-2,-1,...,10}
	\draw (1pt,\x) -- (-1pt,\x) node[anchor=east] {\x};
\draw[thick,color=royalblueweb] plot[samples=4,domain=0:4] (\x,{10-3*(\x)});
\draw[thick,color=royalblueweb] plot[samples=4,domain=0:6] (\x+4,{-2+1*(\x)});
\fill[color=royalblueweb] (0,10) circle (3pt);
\draw[color=royalblueweb] (0,10) node[right] {A};
\fill[color=royalblueweb] (4,-2) circle (3pt);
\draw[color=royalblueweb] (4,-2) node[below] {B};
\fill[color=royalblueweb] (10,4) circle (3pt);
\draw[color=royalblueweb] (10,4) node[above] {C};
\draw[very thick,color=burntorange] (4,4) -- (10,4);
\draw[very thick,color=black,dashed] (4,-2) -- (4,4);
\end{tikzpicture} 
\end{column}
\begin{column}{.4\linewidth}
\LARGE
Discuss with your partner: What is the interpretation of the burnt orange horizontal line segment?  Does the length have a meaning?
\end{column}
\end{columns}
\end{frame}

\begin{frame}{Measuring vertically on a graph, 1}
\begin{columns}[T]
\begin{column}{.6\linewidth}
\begin{tikzpicture}[scale=0.5]
\draw[very thin, color=silver] (-.5,-2.2) grid (10.2,10.2);
\draw[->](-.5,0) -- (10.2,0) node[right] {$t\unit{(s)}$};
\draw[->](0,-2.2) -- (0, 10.2) node[above] {$x\unit{(m)}$};
\foreach \t in {0,1,...,10}
	\draw (\t,1pt) -- (\t,-1pt) node[anchor=north] {\t};
\foreach \x in {-2,-1,...,10}
	\draw (1pt,\x) -- (-1pt,\x) node[anchor=east] {\x};
\draw[thick,color=royalblueweb] plot[samples=4,domain=0:4] (\x,{10-3*(\x)});
\draw[thick,color=royalblueweb] plot[samples=4,domain=0:6] (\x+4,{-2+1*(\x)});
\fill[color=royalblueweb] (0,10) circle (3pt);
\draw[color=royalblueweb] (0,10) node[right] {A};
\fill[color=royalblueweb] (4,-2) circle (3pt);
\draw[color=royalblueweb] (4,-2) node[below] {B};
\fill[color=royalblueweb] (10,4) circle (3pt);
\draw[color=royalblueweb] (10,4) node[above] {C};
\draw[very thick,color=burntorange] (0,4) -- (0,-2);
\draw[very thick,color=black,dashed] (0,4) -- (10,4);
\draw[very thick,color=black,dashed] (0,-2) -- (4,-2);
\end{tikzpicture} 
\end{column}
\begin{column}{.4\linewidth}
\LARGE
Discuss with your partner: What is the interpretation of the burnt orange vertical line segment?  Does the length have a meaning?
\end{column}
\end{columns}
\end{frame}

\begin{frame}{Measuring vertically on a graph, 2}
\begin{columns}[T]
\begin{column}{.6\linewidth}
\begin{tikzpicture}[scale=0.5]
\draw[very thin, color=silver] (-.5,-2.2) grid (10.2,10.2);
\draw[->](-.5,0) -- (10.2,0) node[right] {$t\unit{(s)}$};
\draw[->](0,-2.2) -- (0, 10.2) node[above] {$x\unit{(m)}$};
\foreach \t in {0,1,...,10}
	\draw (\t,1pt) -- (\t,-1pt) node[anchor=north] {\t};
\foreach \x in {-2,-1,...,10}
	\draw (1pt,\x) -- (-1pt,\x) node[anchor=east] {\x};
\draw[thick,color=royalblueweb] plot[samples=4,domain=0:4] (\x,{10-3*(\x)});
\draw[thick,color=royalblueweb] plot[samples=4,domain=0:6] (\x+4,{-2+1*(\x)});
\fill[color=royalblueweb] (0,10) circle (3pt);
\draw[color=royalblueweb] (0,10) node[right] {A};
\fill[color=royalblueweb] (4,-2) circle (3pt);
\draw[color=royalblueweb] (4,-2) node[below] {B};
\fill[color=royalblueweb] (10,4) circle (3pt);
\draw[color=royalblueweb] (10,4) node[above] {C};
\draw[very thick,color=burntorange] (10,10) -- (10,4);
\draw[very thick,color=black,dashed] (0,10) -- (10,10);
\end{tikzpicture} 
\end{column}
\begin{column}{.4\linewidth}
\LARGE
Discuss with your partner: What is the interpretation of the burnt orange vertical line segment?  Does the length have a meaning?
\end{column}
\end{columns}
\end{frame}

\begin{frame}{Measuring diagonally on a graph, 1}
\begin{columns}[T]
\begin{column}{.6\linewidth}
\begin{tikzpicture}[scale=0.5]
\draw[very thin, color=silver] (-.5,-2.2) grid (10.2,10.2);
\draw[->](-.5,0) -- (10.2,0) node[right] {$t\unit{(s)}$};
\draw[->](0,-2.2) -- (0, 10.2) node[above] {$x\unit{(m)}$};
\foreach \t in {0,1,...,10}
	\draw (\t,1pt) -- (\t,-1pt) node[anchor=north] {\t};
\foreach \x in {-2,-1,...,10}
	\draw (1pt,\x) -- (-1pt,\x) node[anchor=east] {\x};
\draw[thick,color=royalblueweb] plot[samples=4,domain=0:4] (\x,{10-3*(\x)});
\draw[thick,color=royalblueweb] plot[samples=4,domain=0:6] (\x+4,{-2+1*(\x)});
\fill[color=royalblueweb] (0,10) circle (3pt);
\draw[color=royalblueweb] (0,10) node[right] {A};
\fill[color=royalblueweb] (4,-2) circle (3pt);
\draw[color=royalblueweb] (4,-2) node[below] {B};
\fill[color=royalblueweb] (10,4) circle (3pt);
\draw[color=royalblueweb] (10,4) node[above] {C};
\draw[very thick,color=burntorange] (2,4) -- (4,-2);
\draw[very thick,color=burntorange] (0,10) -- (10,4);
\end{tikzpicture} 
\end{column}
\begin{column}{.4\linewidth}
\LARGE
Discuss with your partner: What are the interpretations of the burnt orange diagonal line segments?  Does the length have a meaning?
\end{column}
\end{columns}
\end{frame}

\begin{frame}{Measuring diagonally on a graph, 2}
\begin{columns}[T]
\begin{column}{.6\linewidth}
\begin{tikzpicture}[scale=0.5]
\draw[very thin, color=silver] (-.5,-2.2) grid (10.2,10.2);
\draw[->](-.5,0) -- (10.2,0) node[right] {$t\unit{(s)}$};
\draw[->](0,-2.2) -- (0, 10.2) node[above] {$x\unit{(m)}$};
\foreach \t in {0,1,...,10}
	\draw (\t,1pt) -- (\t,-1pt) node[anchor=north] {\t};
\foreach \x in {-2,-1,...,10}
	\draw (1pt,\x) -- (-1pt,\x) node[anchor=east] {\x};
\draw[thick,color=royalblueweb] plot[samples=4,domain=0:4] (\x,{10-3*(\x)});
\draw[thick,color=royalblueweb] plot[samples=4,domain=0:6] (\x+4,{-2+1*(\x)});
\fill[color=royalblueweb] (0,10) circle (3pt);
\draw[color=royalblueweb] (0,10) node[right] {A};
\fill[color=royalblueweb] (4,-2) circle (3pt);
\draw[color=royalblueweb] (4,-2) node[below] {B};
\fill[color=royalblueweb] (10,4) circle (3pt);
\draw[color=royalblueweb] (10,4) node[above] {C};
\draw[very thick,color=black,dashed] (0,10) -- (4,10) -- (4,-2) -- (10,-2) -- (10,4);
\end{tikzpicture} 
\end{column}
\begin{column}{.4\linewidth}\Large
Discuss with your partner: 
\begin{enumerate}
\item How fast is the object moving from A to B?
\item How fast is the object moving from B to C?
\item How fast is the object moving from A to C?
\end{enumerate}
\end{column}
\end{columns}
\end{frame}

\begin{frame}{Intersections}
\begin{columns}[T]
\begin{column}{.6\linewidth}
\begin{tikzpicture}[scale=0.5]
\draw[very thin, color=silver] (0,0) grid (10.2,10.2);
\draw[->](-.5,0) -- (10.2,0) node[right] {$t\unit{(s)}$};
\draw[->](0,0) -- (0, 10.2) node[above] {$x\unit{(m)}$};
\foreach \t in {0,1,...,10}
	\draw (\t,1pt) -- (\t,-1pt) node[anchor=north] {\t};
\foreach \x in {0,...,10}
	\draw (1pt,\x) -- (-1pt,\x) node[anchor=east] {\x};
\draw[thick,color=royalblueweb] plot[samples=4,domain=0:2] (\x,{10-.5*(\x)}) node[right] {object A};
\definecolor{amethyst}{rgb}{0.6,0.4,0.8}
\draw[thick,color=amethyst] plot[samples=4,domain=0:1] (\x,{0+2*(\x)}) node[right] {object B};
\end{tikzpicture} 
\end{column}
\begin{column}{.4\linewidth}\LARGE
Discuss with your partner: At what clock reading and position will the two objects intersect if they continue to move as drawn?
\end{column}
\end{columns}
\end{frame}

\begin{frame}{An interesting graph}
\begin{columns}[T]
\begin{column}{.6\linewidth}
\begin{tikzpicture}[scale=0.5]
\draw[very thin, color=silver] (-.5,-2.2) grid (10.2,10.2);
\draw[->](-.5,0) -- (10.2,0) node[right] {$t\unit{(s)}$};
\draw[->](0,-2.2) -- (0, 10.2) node[above] {$x\unit{(m)}$};
\foreach \t in {0,1,...,10}
	\draw (\t,1pt) -- (\t,-1pt) node[anchor=north] {\t};
\foreach \x in {-2,-1,...,10}
	\draw (1pt,\x) -- (-1pt,\x) node[anchor=east] {\x};
\draw[thick,color=royalblueweb] plot[samples=50,domain=0:8] ({9*(1-(\x/4-1)^2)},\x);
\end{tikzpicture} 
\end{column}
\begin{column}{.4\linewidth}\LARGE
Discuss with your partner: How would you interpret this graph of position versus clock reading?
\end{column}
\end{columns}
\end{frame}

\begin{frame}{Stopped Car Problem}
\begin{columns}[T]
\begin{column}{.6\linewidth}
\begin{tikzpicture}[scale=0.5]
\draw[very thin, color=silver] (0,-3) grid (10,10);
\draw[->](0,0) -- (10.2,0) node[right] {$t\unit{(s)}$};
\draw[->](0,-3) -- (0,10.2) node[above] {$x\unit{(m)}$};
\foreach \t in {0,1,...,10}
	\draw (\t,1pt) -- (\t,-1pt) node[anchor=north] {\t};
\draw (1pt,0) -- (-1pt,0) node[anchor=east] {0};
\foreach \x in {1,2,...,10}
	\draw (1pt,\x) -- (-1pt,\x) node[anchor=east] {\x0};
\foreach \x in {-1,-2,-3}
	\draw (1pt,\x) -- (-1pt,\x) node[anchor=east] {\x0};
\definecolor{amethyst}{rgb}{0.6,0.4,0.8}
\draw[very thick,color=amethyst] plot[samples=40,domain=0:9] (\x,{.5*2.5*(\x)^2/10}) node[above] {other car};
\draw[very thick,color=royalblueweb,dashed] plot[samples=50,domain=0:6.5] (\x,{-3+20*\x/10}) node[left] {your car};
\end{tikzpicture} 
\end{column}
\begin{column}{.4\linewidth}\large
Discuss with your partner: 
You are driving a car approaching a stoplight that has just turned green.  A car, which was stopped at the stoplight, is just starting to move but is not yet up to speed.  If you continue at your present speed, you will have to brake hard at the last moment lest you run into the other car.  What should you do so that you have to slow down as little as possible?
%To make a connection between our everyday experience and meters per second, it is useful to know that 10 mph is the same as 4.4704 meters per second.  A car traveling at 20 meters per second is traveling about 45 mph.  The acceleration of the other car was chosen to be 2.5 meters per second per second.
\end{column}
\end{columns}
\end{frame}

\begin{frame}{Practice}\Huge
For homework, take ``\textbf{Unit~1: Uniform~Motion Worksheet~3}''.  Bring it completed to the next class.
\end{frame}
\end{document}
